\documentclass{article}

\usepackage{verbatim}
\usepackage{xspace}

\usepackage[unicode]{hyperref}

\usepackage{pstricks}
\usepackage{pst-node}           % PSTricks package for nodes
\usepackage{pst-tree}           % PSTricks package for trees
%\usepackage{pst-3dplot}
%\usepackage{amsmath} % ��� cfrac
\psset{linewidth=0.5pt,arrowlength=2.8}


\newcommand{\Skeleton}{{\sc Skeleton}\xspace}
\newcommand{\SkeletonVersion}{{\sc 02.01.03}\xspace}
\newcommand{\SkeletonDate}{October 19, 2010}
\newcommand{\params}{\smallskip\hangindent=2\parindent}
\newcommand{\Arageli}{{\sc Arageli}\xspace}

\newtheorem{theorem}{Theorem}
\newtheorem{lemma}{Lemma}
\newtheorem{proposition}{Proposition}

\newcommand{\NN}{{\cal N}}
\newcommand{\BB}{{\cal B}}
\newcommand{\LL}{{\cal L}}
\newcommand{\LD}{{\cal LD}}
\newcommand{\PP}{{\cal P}}
\newcommand{\RRR}{{\cal R}}
\newcommand{\SSS}{{\cal S}}
\newcommand{\ZZ}{{\rm\bf Z}}
\newcommand{\QQ}{{\rm\bf Q}}
\newcommand{\RR}{{\rm\bf R}}
\newcommand{\VV}{{\rm\bf V}}
\newcommand{\CC}{{\rm\bf C}}
\newcommand{\ee}{{\rm\bf e}}
\newcommand{\ff}{{\rm\bf f}}
\newcommand{\FF}{{\cal F}}
\renewcommand{\Re}{\mathop{\rm Re}\nolimits}
\renewcommand{\Im}{\mathop{\rm Im}\nolimits}
\newcommand{\VS}{\mathop{\rm VS}\nolimits}
\newcommand{\dist}{\mathop{\rm dist}\nolimits}
\newcommand{\Aff}{\mathop{\rm Aff}\nolimits}
\newcommand{\Lin}{\mathop{\rm Lin}\nolimits}
\newcommand{\NonNeg}{\mathop{\rm NonNeg}\nolimits}
\newcommand{\Conv}{\mathop{\rm Conv}\nolimits}
\newcommand{\Ker}{\mathop{\rm Ker}\nolimits}
\newcommand{\rank}{\mathop{\rm rank}\nolimits}
\newcommand{\pr}{\mathop{\rm pr}\nolimits}
\newcommand{\ort}{\mathop{\rm ort}\nolimits}
\newcommand{\set}[1]{\left\{ #1\right\}}
\newcommand{\oset}[1]{\left\langle #1\right\rangle}
%\newcommand{\intie}[1]{\left[ #1\right]}
\newcommand{\intie}[1]{\left\lfloor #1\right\rfloor}
\newcommand{\intieup}[1]{\left\lceil #1\right\rceil}
\newcommand{\transpose}{^{\top}}
\newcommand{\GCD}{\mathop{\rm ���}\nolimits}
\newcommand{\LCC}{\mathop{\rm ���}\nolimits}
\newcommand{\diag}{\mathop{\rm diag}}
\newcommand{\defect}{\mathop{\rm def}\nolimits}
\newcommand{\degree}{\mathop{\rm deg}\nolimits}
\newcommand{\pp}{\stackrel{.}{+}}
\newcommand{\Sum}{\sum\limits}
\newcommand{\tr}{\mathop{\rm tr}\nolimits}
\newcommand{\lexmin}{\mathop{\rm lexmin}\limits}
\newcommand{\comments}[1]{}
\newcommand{\titlebf}[1]{\par\noindent{\bf #1.}}
\newcommand{\titlesc}[1]{\par\noindent{\sc #1.}}
\newcommand{\fracc}[2]{\frac{\textstyle #1\strut}{\textstyle #2\strut}}
\newcommand{\fraccc}[2]{\leavevmode\kern.1em\raise.5ex\hbox{\scriptsize $#1$}
  \kern-.1em / \kern-.15em \lower.25ex\hbox{\scriptsize $#2$}\,}
\newcommand{\fr}[1]{\fbox{$#1$}}
\newcommand{\newsubsection}{\medskip\centerline{* * *}\medskip\nopagebreak}
\newenvironment{narrowarray}[1]{\arraycolsep=0.15em\begin{array}{#1}}{\end{array}}
\newcommand{\lpp}[2]{\begin{array}{c}#1\\[8pt]{} #2\end{array}}
\newcommand{\no}[1]{^{(#1)}}
\newcommand{\tpsb}[2]{\begin{MAT}(e){cc} #1 & \star \\ & #2 \\ \end{MAT}}
\newcommand{\tpsn}[1]{\begin{MAT}(e){cc} #1 & \\ & \\    \end{MAT}}
\newcommand{\Ref}[1]{{\rm (\ref{#1})}}
\newcommand{\xhat}{{\widehat x}}
\newcommand{\xtilde}{{\widetilde x}}
\newcommand{\xbar}{{\overline{x}}}
\newcommand{\uhat}{{\widehat u}}
\newcommand{\utilde}{{\widetilde u}}
\newcommand{\ahat}{{\widehat a}}
\newcommand{\atilde}{{\widetilde a}}
\newcommand{\bhat}{{\widehat b}}
\newcommand{\btilde}{{\widetilde b}}
\newcommand{\chat}{{\widehat c}}
\newcommand{\fhat}{{\widehat f}}
\newcommand{\ctilde}{{\widetilde c}}
\newcommand{\vtilde}{{\widetilde v}}
\newcommand{\wtilde}{{\widetilde w}}
\newcommand{\ytilde}{{\widetilde y}}
\newcommand{\ttilde}{{\widetilde t}}
\newcommand{\Btilde}{{\widetilde B}}
\newcommand{\Ntilde}{{\widetilde N}}

\newcounter{formula}
%\newcounter{formula}[chapter]
\renewcommand{\theformula}{\arabic{formula}}
%\renewcommand{\theformula}{\thechapter.\arabic{formula}}
%\newcommand{\tag}[1]{\refstepcounter{formula}
%  \eqno{(\theformula:#1)}\label{#1}}
\newcommand{\tag}[1]{\refstepcounter{formula}
  \eqno{(\theformula)}\label{#1}}


\begin{document}

\title{\Skeleton \SkeletonVersion Manual}

\date{\SkeletonDate}

\author{Nikolai Yu. Zolotykh\\[0.5em]
\small with participation of Aleksey Bader, Sergey Lobanov, Sergey Lyalin\\[1em]
 N.I.\,Lobachevsky State University of Nizhni Novgorod, Russia}

\maketitle

\begin{abstract}
This paper describes \Skeleton: implementation of several new variations of
well-known Double Description Method (DDM) for solving the vertex and facet
enumeration problems for convex polyhedra. New enhancements makes \Skeleton 
quite competitive in comparison with other implementations of DDM.
The source code of \Skeleton \SkeletonVersion is available at
\url{http://www.uic.nnov.ru/~zny/skeleton}.
\end{abstract}

\tableofcontents

\newpage

\newcommand{\AB}{(A\transpose,B\transpose)}%{\left(\begin{array}{c}{A\\ B}\end{array}\right)}

\section{What's new?}
\ 

\params \Skeleton 02.00.00 May 7, 2006\\ It is new, completely re-written, fast version of \Skeleton.

\params \Skeleton 02.00.01 November 1, 2006\\
\Skeleton now runs on Linux platform. Source code is available.

\params \Skeleton 02.00.02 November 7, 2006\\
Floating point arithmetic is now supported.

\params \Skeleton 02.00.03 May 30, 2007\\
Time bug fixed.
%\item edges
%Vertices of 2d and 3d polyhedra are in clockwise or anticlockwise direction.

\params \Skeleton 02.00.04 October 6, 2009\\
A bug occuring on 64 bit architecture fixed. (Thanks to Sergey Lyalin and Sergey Lobanov.)

\params \Skeleton 02.01.00 November 16, 2009:
\begin{itemize}
\item The possibility to explicitly set a system of linear equations in addition to a system of linear inequalities.
      See \verb$--skeletonformat$ option.
\item New option \verb$--ridges$ for constructing ridges.
\item New option \verb$--facetadjacency$ for constructing lists of adjacent facets.
\item New option \verb$--verifyine$ to determine implicit equations and redundand inequalities in the input system.
\item Avis--Fukuda format is now (partially) supported. See \verb$--avisfukudaformat$ option.
\item New option \verb$--silence$ is available.      
\item New options \verb$--inputfromstdin$, \verb$--noinputfromstdin$ are now available.
\item According GNU style in long option names double hyphen is used instead of single hypen, 
      for example, \verb$--minindex$ instead of \verb$-minindex$.
\item \verb$-graphinc$ option is renamed \verb$--graphadj$ (from {\it graph of potential adjacency}).
\item \verb$-inc$ option is renamed \verb$--dis$ (from {\it discrepancies}).
\item \verb$-minedges$, \verb$-maxedges$ options are renamed \verb$--minpairs$ and \verb$--maxpairs$
      correspondingly.
\item \verb$-incext$, \verb$-noincext$, \verb$-incine$, \verb$-noincine$ options are renamed 
      \verb$--extinc$, \verb$--noextinc$, \verb$--ineinc$, \verb$--noineinc$ correspondingly.
\end{itemize}

\params \Skeleton 02.01.01 July 11, 2010\\
      Installation procedure became simpler.
      Make file (for Linux) and project file (for MS Visual Studio) are provided now.
      \Arageli is in the \Skeleton distribution now and you need not install \Arageli separately.
      A lot of thanks to Aleksey Bader and Sergey Lyalin.

\params \Skeleton 02.01.02 July 15, 2010\\
      An installation bug is fixed.
      Thanks to Aleksey Bader, Sergey Lyalin and Sergey Lobanov.
      Now in this manual instead of term {\em skeleton} (not program \Skeleton) I use
      the term {\em ossature}.

\params \Skeleton 02.01.03 October 19, 2010\\
      A bug connected with division by GCD when using skeleton format is fixed.


\section{Introduction}

It is well known that any polyhedron in $\RR^d$
can be represented by the following two ways:
\begin{enumerate}
  \item[(1)] as a set of solutions to the system of linear inequalities, or

  \item[(2)] as the (Minkowski's) sum of the conic hull of some vectors and 
                           the convex hull of some points in $\RR^d$.
\end{enumerate}

The problem to generate representation (2) if representation (1) is available
is called the {\em vertex enumeration problem}. The converse one
is called the {\em facet enumeration problem}, or {\em convex hull problem}.

Analogously, any polyhedral cone in $\RR^d$
can be represented by the following two ways:
\begin{enumerate}
  \item[(1)] as a set of solutions to the system of homogenius linear inequalities, or
  \item[(2)] as a set of all non-negative linear combinations of some vectors in $\RR^d$.
\end{enumerate}

There is a standard way to reduce vertex/facet enumeration problem for polyhedra
to the correspondent problem for polyhedral cones. From theoretical point of view
it is convenient to consider both problems just for polyhedral cones. 

The program \Skeleton implements several variations of Double Description
Method (DDM) \cite{MRTT53} solving the vertex and facet enumeration
problems. DDM is considered in a few papers and monographs 
\cite{Burger56,Chernikova64,Chernikova65,Chernikov68,VPSh84,Chernikova68,FQ88,Verge92,FP96,ShCh97,ShG2003}.

\Skeleton works with the system of linear inequalities whose entries
are integers (arbitrary precision or $4$ bytes long ints) or reals
(double floating point numbers). 

In our implementation we
use ideas descibed in \cite{VPSh84,FP96,ShCh97} and some new enhancements.
All these makes \Skeleton quite competitive in comparison with other
implementations of DDM, in particular, \cite{Verge92,Fukuda02,Gruzdev2003}. Early version of
\Skeleton is described in \cite{Zolotykh97}.

\Skeleton can be distributed under the terms of GNU GENERAL PUBLIC LICENSE Version 2. 
Read file COPYING.

Thanks to Sergey Lobanov you can use \Skeleton on-line (without install it).
Visit \url{http://www.arageli.org}.



\section{Theoretical Preliminaries}

\subsection{Polyhedral Cones}

{\em Polyhedral cone} $C$ is the set of all solutions to a system of 
homogenius linear inequalities and equations $Ax\ge 0$, $Bx= 0$:
$$
C = \set{x \in \RR^d:~ Ax \ge 0,~ Bx = 0},   \eqno(\star)
$$
where $A \in \RR^{m \times d}$, $B \in \RR^{t \times d}$
($m$ or/and $t$ may be equal to $0$ that corresponds to the case when
inequalities or/and equations are absent accordingly).
The case of system of equations and inequalities can be obviously reduced 
to the case of system with only inequalities. For this instead of $Bx = 0$ 
we can consider $Bx\ge 0$ and $-Bx\ge 0$. But it will be more convinient to
consider the more general case.

The maximal subspace contained in the cone $C$ can be
described as a set of all solutions to the system $Ax = 0$, $Bx = 0$. The dimension
of this subspace is equal to $d - \rank \AB$. The cone is called {\em
pointed} if it contains only zero subspace, that's equivalent to $\rank \AB = d$.

Let $a\in\RR^d$, $a\ne 0$.
The hyper-plane $\set{x\in\RR^d:~ ax=0}$ is called {\em supporting} for the cone $C$
if %$C\cap\set{x:~ ax= 0}\ne\emptyset$ and 
$C\subseteq\set{x:~ ax\ge 0}$ or
$C\subseteq\set{x:~ ax\le 0}$. The intersection of the cone with a supporting
hyper-plain is called its {\em face}.

Consider the system of vectors $u_1,\dots,u_p$ in $\RR^d$. 
The {\em linear hull} $\Lin(u_1,\dots,u_p)$ of the system is the set 
of all linear combinations of these
vectors:
$$
\Lin(u_1,\dots,u_p) = 
\set{\lambda_1 u_1 + \dots + \lambda_p u_p:~ \lambda_i \in \RR~ (i=1,\dots,p)}.
$$
The {\em non-negative}, or {\em conic}, {\em hull} $\NonNeg(u_1,\dots,u_p)$ 
of the system is the set 
of all non-negative linear combinations:
$$
\NonNeg(u_1,\dots,u_p) = \set{\lambda_1 u_1 + \dots + \lambda_p u_p:~ 
\lambda_i \in \RR,~\lambda_i \ge 0~ (i=1,\dots,p)}.
$$

\begin{theorem}[Minkowski]
For any polyhedral cone $C$ in $\RR^d$ there exist vectors
$u_1,\dots,u_p$, $v_1,\dots,v_q$ in $\RR^d$ such that
$$
C=\Lin(u_1,\dots,u_p) + \NonNeg(v_1,\dots,v_q).  \eqno(\star\star)
$$
\end{theorem}

Obviously, w.l.g. in Minkowski's theorem 
we can omit $\Lin(u_1,\dots,u_p)$ item and instead of $(\star\star)$ 
write simply $C=\NonNeg(v_1,\dots,v_q)$.

Using matrix notation we can re-formulate Minkowski's theorem as follows.
For any matrices $A \in \RR^{m\times d}$ and $B \in \RR^{t\times d}$ there exist matrices 
$V\in \RR^{q\times d}$ and $U\in \RR^{p\times d}$ such that
$$
\set{x\in\RR^d:~Ax\ge 0, ~ Bx = 0} = 
\set{x = \mu V + \lambda U :~ \mu \in \RR^n,~ \mu \ge 0,~ \lambda \in \RR^s}.
$$

Vectors $u_1,\dots,u_p$, $v_1,\dots,v_q$ (equvalently,
matrices $U$ and $V$) can be chosen in such a way that 
the following {\em properties of minimality} hold:
\begin{enumerate}
  \item $p = d - \rank \AB$
        and $\Lin(u_1,\dots,u_p)$ is the maximal subspace $L$ included in $C$
        (so, the system $u_1,\dots,u_p$ is a basis of $L$); and
  \item $q$ is minimal among all possible $q$ such that ($\star\star$)
        holds (this means also that the system $v_1,\dots,v_q$ is irreducible); 
        in this case the system $v_1,\dots,v_q$ is called an {\em ossature}
        of the cone $C$.
\end{enumerate}
The vectors in an ossature are unique up to any positive multiplyer and any item in $L$.
If the cone $C$ is pointed, i.e. $\rank \AB = d$ and, hence, $p=0$, then $v_1,\dots,v_q$ 
(and --- up to positive multiplyer --- only they)
are {\em extreme rays} of $C$. We'll say that two vectors in an ossature are
{\em adjacent} if minimal face containing both does not contain any other vector in the ossature.

The converse theorem to Minkowski's one is correct and is known as Weyl's theorem.

\begin{theorem}[Weyl]
For any vectors
$u_1,\dots,u_p$, $v_1,\dots,v_q$ in $\RR^d$ there exist matrices
$A\in \RR^{m\times d}$ and $B\in \RR^{t\times d}$ such that
$$
\Lin(u_1,\dots,u_p) + \NonNeg(v_1,\dots,v_q) = \set{x\in\RR^d:~Ax\ge 0,~Bx= 0}.
$$
\end{theorem}

Obviously, w.l.g. in Weyl's theorem 
we can omit linear equations $Bx=0$ and 
simply write $\Lin(u_1,\dots,u_p) + \NonNeg(v_1,\dots,v_q) = \set{x\in\RR^d:~Ax\ge 0}$.

Using matrix notation we can re-formulate Weyl's theorem as follows.
For any matrices 
$U\in \RR^{p\times d}$ and $V\in \RR^{q\times d}$ 
there exist matrices $A \in \RR^{m\times d}$ and $B\in \RR^{t\times d}$ such that
$$                 
\set{x = \lambda U + \mu V:~ \lambda \in \RR^p,~ \mu \in \RR^q,~ \mu \ge 0} =
\set{x\in\RR^d:~Ax\ge 0,~Bx= 0}.
$$

Matrices $A$ and $B$ can be chosen in such a way that 
the following {\em properties of minimality} hold:
\begin{enumerate}
  \item $t = d - \rank \set{u_1,\dots,u_p, v_1,\dots,v_q}$
        and $\set{x: Bx = 0}$ is the minimal subspace containing $C$
        (this means also that the system $Bx = 0$ is irreducible); and
  \item $m$ is minimal among all possible $m$ such that ($\star$)
        holds (this means also that the system $Ax \ge 0$ is irreducible).
\end{enumerate}
The rows in such a matrix $A$ are unique up to any positive multiplyer and
any item which is linear combinations of rows in $B$. The rows in $A$
correspond to faces of maximum dimension. In particular, if the cone $C$ is
full-dimensional, i.e. $\rank \set{u_1,\dots,u_p, v_1,\dots,v_q} = d$ and,
hence, $t=0$, then the rows in $A$ correspond to {\em facets} of $C$.

These theorems suggest two fundamental problems. First one is to obtain a dual representation
($\star\star$) if a representation ($\star$) is known. The second problem is converse.
It turns out that these problems are computationaly equivalent as the following theorem shows.
So, we can concentate on the first problem.

\begin{theorem}[Farkas--Minkowski--Weyl]
If 
$$
C=\set{x\in\RR^d:~Ax\ge 0, ~ Bx = 0} = 
\set{x = \mu V + \lambda U :~ \mu \in \RR^q,~ \mu \ge 0,~ \lambda \in \RR^p}
$$
then
$$
C'=\set{x\in\RR^d:~Vx\ge 0, ~ Ux = 0} = 
\set{x = \mu A + \lambda B :~ \mu \in \RR^m,~ \mu \ge 0,~ \lambda \in \RR^t}.
$$
\end{theorem}

Moreover, if rows in $V$ and $U$ form an ossature of $C$ and a basis of minimal subspace
correspondingly, then $Vx\ge 0$, $Ux = 0$ are irreducible systems determining
$C'$ and viceversa. Analogous property is true for $A$ and $B$.




\subsection{Polyhedra}

{\em Polyhedron} $P$ is the set of all solutions to a system of 
linear inequalities and equations $Ax\ge b$, $Bx= c$:
$$
P = \set{x \in \RR^d:~ Ax \ge b,~ Bx = c},  \eqno(*)
$$
where $A \in \RR^{m \times d}$, $B \in \RR^{t \times d}$, $b\in \RR^{m}$, $c\in \RR^t$
($m$ or/and $t$ may be equal to $0$ that corresponds to the case when
inequalities or/and equations are absent accordingly).
The case of system of equations and inequalities can be obviously reduced 
to the case of system with only inequalities. But it will be more convinient to
consider the more general case.

Let $a\in\RR^d$, $a\ne 0$, $\alpha \in \RR$.
The hyper-plane $\set{x\in\RR^d:~ ax=\alpha}$ is called {\em supporting} for the polyhedron $P$
if $P\cap\set{x:~ ax= \alpha}\ne\emptyset$ and 
$P\subseteq\set{x:~ ax\ge \alpha}$ or
$P\subseteq\set{x:~ ax\le \alpha}$. The intersection of the polyhedron with a supporting
hyper-plane is called a {\em face} of the polyhedron. The face whith dimension $0$ 
(i.e. a point) is called a {\em vertex} of $P$.


Consider the system of vectors $w_1,\dots,w_s$ in $\RR^d$. 
%The {\em affine hull} $\Aff(u_1,\dots,u_s)$ of the system is the set 
%of all {\em affine combinations} of these vectors, i.e.:
%$$
%\Aff(u_1,\dots,u_s) = 
%\set{\lambda_1 u_1 + \dots + \lambda_s u_s:~ \lambda_i \in \RR,~ \sum_{i=1}^s\lambda_i=1}.
%$$
The {\em convex hull} $\Conv(w_1,\dots,w_s)$ of the system is the set 
of all {\em convex combinations} of these vectors, i.e.:
$$
\Conv(w_1,\dots,w_s) = 
\set{\lambda_1 w_1 + \dots + \lambda_s w_s:~ \lambda_i \in \RR,~ \lambda_i \ge 0,~
\sum_{i=1}^s\lambda_i=1}.
$$
The set of points in $\RR^d$ which can be represented as a convex hull of some
finite system of points is called {\em polytope}.

From Minkowski's theorem we get the following.

\begin{theorem}
For any polyhedron $P$ in $\RR^d$ there exist vectors
$u_1,\dots,u_p$, $v_1,\dots,v_q$, $w_1,\dots,w_s$ in $\RR^d$ such that
$$
P=\Lin(u_1,\dots,u_p) + \NonNeg(v_1,\dots,v_q) + \Conv(w_1,\dots,w_s). \eqno(**)
$$
So, any polyhedron is the sum of a cone and a polytope.
\end{theorem}

Obviously, w.l.g. in the theorem 
we can omit $\Lin(u_1,\dots,u_p)$ item and instead of $(**)$ 
write simply $P=\NonNeg(v_1,\dots,v_q)+ \Conv(w_1,\dots,w_s)$.


Vectors $u_1,\dots,u_p$, $v_1,\dots,v_q$, $w_1,\dots,w_s$ 
can be chosen in such a way that 
the following {\em properties of minimality} hold:
\begin{enumerate}
  \item $u_1,\dots,u_p$ is a basis of the subspace $L$ associated with
        the maximal linear variety in $P$; and
  \item $q$ and $s$ are minimal among all possible $q$ and $s$ such that ($**$)
        holds (this means also that the systems $v_1,\dots,v_q$ and
        $w_1,\dots,w_s$ are irreducible).
\end{enumerate}
In this case 
vectors $w_1,\dots,w_s$ are unique up to any item in $L$;
vectors $v_1,\dots,v_q$ are unique up to any positive multiplyer and any item in $L$.
If $p=0$ then points $w_1,\dots,w_s$ (and only they) are vertices of $P$
          and vectors $v_1,\dots,v_q$ (and only they) are extreme rays of $P$.


The problem of constructing the representation $(**)$ if representation $(*)$  is available
is called
the {\em vertex enumeration problem}. 
It can be reduced to the analogous problem for cones as follows.

Consider the cone in $\RR^{d+1}$
$$
C = \set{(x_1,\dots,x_n,x_{n+1})\transpose \in \RR^{d+1}:
~ Ax\ge bx_{n+1}, 
~ Bx=cx_{n+1},
~x_{n+1} \ge 0},
$$
where $x=(x_1,\dots,x_n)\transpose$.
For the cone $C$ we can get a dual representation
$$
C=\Lin(\overline{u}_1,\dots,\overline{u}_p) + \NonNeg(\overline{v}_1,\dots,\overline{v}_q)
$$
for some vectors 
$\overline{u}_1,\dots,\overline{u}_p$, 
$\overline{v}_1,\dots,\overline{v}_q$ in $\RR^{d+1}$.

Let 
$\overline{u}_i=(u_i,u_{i,n+1})$ $(i=1,\dots,p)$, 
$\overline{v}_i=(v_i,v_{i,n+1})$ $(i=1,\dots,q)$.
Since the system of homogenius linear inequalities and equations contains the inequality
$x_{n+1} \ge 0$, then it is clear that $u_{i,n+1} = 0$ $(i=1,\dots,p)$.
Suppose w.l.g. that 
$v_{i,n+1} = 0$ $(i=1,\dots,s)$, $v_{i,n+1} \ne 0$ $(i=s+1,\dots,q)$.
Now it is not hard to see that the initial polyhedron $P$ has the following dual representation:
$$
\begin{narrowarray}{ccl}
P & = & \Lin(u_1,\dots,u_p) +   \\[.3em]
  &   &  + \NonNeg(v_1,\dots,v_s) + 
           \Conv\left(\fracc{1}{v_{s+1,d+1}}\cdot v_{s+1},\dots,\fracc{1}{v_{q,d+1}}\cdot v_q\right).
\end{narrowarray}
$$
Moreover, if the system $\overline{u}_1,\dots,\overline{u}_p$ 
is a basis of the maximal linear subspace in $C$ and
the system $\overline{v}_1,\dots,\overline{v}_q$ is an ossature of $C$ then
the system of vectors constructed to describe $P$ also has the property of
minimality. In particular, if $p=0$ then 
$\fracc{1}{v_{s+1,d+1}}\cdot v_{s+1},\dots,\fracc{1}{v_{q,d+1}}\cdot v_q$
are vertices of $P$.

From Weyl's theorem we get the following.

\begin{theorem}
For any vectors
$u_1,\dots,u_p$, $v_1,\dots,v_q$, $w_1,\dots,w_s$,  in $\RR^d$ there exist matrices
$A\in \RR^{m\times d}$ and $B\in \RR^{t\times d}$ and vectors $b\in \RR_m$, $c\in \RR^t$ such that
$$
\begin{array}{r}
\Lin(u_1,\dots,u_p) + \NonNeg(v_1,\dots,v_q) + \Conv(w_1,\dots,w_s)=\qquad\qquad\qquad\\[.3em]
=\set{x\in\RR^d:~Ax\ge b,~Bx=c}.
\end{array}
$$
\end{theorem}


Matrices $A$ and $B$ can be chosen in such a way that 
the following {\em properties of minimality} hold:
\begin{enumerate}
  \item the system $Bx = c$ is irreducible and 
        $\set{x: Bx = c}$ is the minimal linear variety containing $P$; and
  \item $m$ is minimal among all possible $m$ such that ($*$)
        holds (this means also that the system $Ax \ge 0$ is irreducible).
\end{enumerate}
In this case the rows in the matrix $(A, b)$ correspond to faces of maximum dimension. In
particular, if $P$ is full-dimensional, i.e. 
$$\rank \set{u_1,\dots,u_p, v_1,\dots,v_q, w_1 - w_s,\dots,w_{s-1}-w_s} = d$$ 
and, hence, $t=0$, then the rows in $(A, b)$ 
correspond to {\em facets} of $P$.

The problem of constructing the representation $(*)$ if representation $(**)$ is available
is called
the {\em facet enumeration problem}, or the {\em convex hull problem}. 
It can be reduced to the analogous problem for cones as follows.

In $\RR^{d+1}$ consider the cone
$$
C=
\Lin(\overline{u}_1,\dots,\overline{u}_p) + 
\NonNeg(\overline{v}_1,\dots,\overline{v}_q, \overline{w}_1,\dots,\overline{w}_s),
$$
where 
$\overline{u}_i = (u_i, 0)$ $(i=1,\dots,p)$, 
$\overline{v}_i = (v_i, 0)$ $(i=1,\dots,q)$, 
$\overline{w}_i = (w_i, 1)$ $(i=1,\dots,s)$ 
and find its representation 
$$
C=\set{(x_1,\dots,x_d,x_{d+1})\transpose\in\RR^{d+1}:~ Ax-bx_{d+1} \ge 0, ~ Bx -cx_{d+1}= 0},
$$
where $x=(x_1,\dots,x_d)\in\RR^d$.
Now it is not hard to see that the initial polyhedron $P$ 
has the following representation:
$$
P=\set{x\in\RR^d:~Ax\ge b,~Bx= c}.
$$
Moreover, if $A$, $B$, $b$, $c$ are such that each of the systems
$Ax-bx_{d+1} \ge 0$ and $Bx = cx_{d+1}$ is irreducible and
$\set{(x_1,\dots,x_d,x_{d+1})\transpose\in\RR^{d+1}:~Bx -cx_{d+1}= 0}$ is
the minimal subspace containing $C$ then the system of inequalities and
equations constructed to describe $P$ also has a property of minimality.



\subsection{The main idea of the algorithm}

Given a matrix $A\in\RR^{m\times d}$, DDM generates
a basis of maximal subspace and an ossature of the cone $C=\set{x\in\RR^d:~Ax\ge 0}$.
Obviously, the case then the cone is defined by a system of linear
inequalities and equations can be reduced to the case with only inequalities.

In the preliminary step of DDM the rank $r$ of $A$ and a basis of the
maximal subspace containing in $C$ are founded. Also, an ossature of the
cone determined by some irreducible subsystem containing $r$ inequalities
is generated. Then, other inequalities are added one after the other and
every time the ossature is re-conctructed. Consider this slightly in
detail.

Let $K$ be a cone determined by some subsystem of $Ax\ge 0$.
Suppose that an ossature of $K$  is known.
Consider what will happen with the ossature then a new inequality $ax \ge 0$ 
is added. 

Each vector in the ossature of $K$
falls to one of the following sets: 
\begin{enumerate}
  \item $W_0$ is the set of all vectors $w$ in the ossature such that $aw = 0$;
  \item $W_+$ is the set of all vectors $w$ in the ossature such that $aw \ge 0$;
  \item $W_-$ is the set of all vectors $w$ in the ossature such that $aw \le 0$.
\end{enumerate}
A ossature of the new cone is formed by all elements in $W_+$ and $W_0$
and vectors which we obtain as follows. For each pair of vectors
$w'\in W_+$ and $w''\in W_-$ adjacent in $K$ we obtain their linear combination $w$
satisfying to equality $aw=0$. Every such $w$ should be included to the ossature
of the new cone.
        
Variations of DDM differs one from another by 
ordering in which inequalities are choose from the system,
the methods used to find adjacent rays,
a time when the adjacency is computed and others \cite{VPSh84,FP96,ShCh97}.
Checking the adjacency seems the most time-expensive procedure in DDM
and different techniques to determine what pairs of vectors should be 
verifying are used \cite{FP96}.

\section{How to Build}

The source code of \Skeleton is available at 
\verb$http://uic.nnov.ru/~zny/skeleton$.
The package contains a documentation, examples, \Arageli library and three C++ files:
\verb$skeleton.cpp$, \verb$ddm.hpp$ and \verb$ddmio.hpp$.

\Skeleton uses \Arageli library \cite{Arageli}.
\Arageli is included in the distribution and it will be compiled automatically.
To use newer version of \Arageli (for example downloaded from
the site \cite{Arageli}), just replace it in the directory
\verb$tools/arageli$.

To compile the code in standard Linux environment you need \verb$gcc$ version 4 or above. Type
\begin{verbatim}
    make
\end{verbatim}
We supposed to be in the root directory of \Skeleton distribution, so after that command, binary
\verb$skeleton$ will appear in the root directory of the distribution.

To compile the code in Windows you can use MS Visual Studio 2008 or later. Please refer to
\verb$msvs_2008$ directory and \verb$skeleton.sln$ solution file (just build entire solution).
Note that for this building way, executables will appear in \verb$bin$ directory and will be named by pattern
\verb$skeleton{32,64}{d,r,f,t}.exe$ corresponding to choosen configuration (32 or 64 bit and
Debug, Release, Fast or Test configuration). The fastest configuration is Fast (\verb$skeleton32f.exe$ or
\verb$skeleton64f.exe$ depending on operating system used), so make sure
that Fast configuration is choosen before building MS Visual Studio solution.


\section{How to Use}

Given a matrix $A\in\ZZ^{m\times d}$, program \Skeleton generates
a basis of maximal subspace and an ossature of the cone $C=\set{x\in\RR^d:~Ax\ge 0}$.

To use \Skeleton, first of all, one should prepare a file with your data. 
The file must contain the size and entries of matrix $A$.
%Entries of $A$ should be integer.
Numbers are separated by spaces and blank lines.
For example, if you want to find an ossature of the cone $C$
defined as a set of solution to the system
$$
\left\{
\begin{narrowarray}{cccccccccl}
 & x_1 &   &     &   &    &   &     & \ge & 0, \\
-& x_1 &   &     &  +& x_3&  +& x_4 & \ge & 0, \\
 &     &  -& x_2 &  +& x_3&   &     & \ge & 0, \\
 &     &   &     &   & x_3&  +& x_4 & \ge & 0, \\
 & x_1 &  +& x_2 &   &    &  +& x_4 & \ge & 0, \\
-& x_1 &  -& x_2 &   &    &  -& x_4 & \ge & 0  \\
\end{narrowarray}
\right.
$$
then the input file (say \verb$example.ine$) is
\verbatiminput{../examples/example.ine}
 
To run \Skeleton just type in the command prompt:
\begin{verbatim}
skeleton filename
\end{verbatim}
where \verb$filename$ is the name of the input file.
Example:
\begin{verbatim}
skeleton example.ine
\end{verbatim}
(The file \verb$example.ine$ and other example files mentioned below is in the folder \verb$examples$.)

\Skeleton produces two files: ``output'' file, ``log'' file
and ``summary'' file. 
By default, their names are obtained by adding
extention \verb$.out$, \verb$.log$, \verb$.sum$  respectively to input file name. 
In our example \Skeleton produces files \verb$example.ine.out$, \verb$example.ine.log$, 
and \verb$example.ine.sum$.

The output file contains sizes and entries of matrix $U$ (vectors of a
basis in row-wise order) and matrix $V$ (vectors of an ossature). Also, the
file can contain other information (it depends on options used; see the
list of available options below). In our example we get the following file
\verb$example.ine.out$: 
\verbatiminput{../examples/example.ine.out}
Thus, we get 
$u_1 = (0, -1, -1, 1)\transpose$, 
$v_1 = (1, -1,  1, 0)\transpose$, 
$v_2 = (0,  0,  1, 0)\transpose$ and
$C=\Lin(u_1) + \NonNeg(v_1, v_2)$.


The log file contains computation hystory.
By default, this information is also displayed on stdcrt during computation. 
In our example we get the following file \verb$example.ine.log$:
\verbatiminput{../examples/example.ine.log}

The summary file contains computation summary.
By default, this information are also displayed on stdcrt after computation. 
In our example we get the following file \verb$example.ine.sum$:
\verbatiminput{../examples/example.ine.sum}

You may set different options affecting the process of computation and
the output of information:
\begin{verbatim}
skeleton filename options
\end{verbatim}
where \verb$options$
is a list of options. Each option is
an abbreviation usually beginning with double hyphen. Options are separated by spaces.
Example:
\begin{verbatim}
skeleton example.ine --lexmin --adjacency
\end{verbatim}
Complete list of all available options is in the next section.



\section{Options}

The following options are available (the default parameters are in braces):

\params
\{\verb$--minindex$\}, \{\verb$--maxindex$\}, \verb$--lexmin$, \verb$--lexmax$, \verb$--random$, \verb$--mincutoff$, 
\verb$--maxcutoff$, \verb$--minpairs$, \verb$--maxpairs$ \\
   These options
   affect the ordering of inequalities to be added at each iteration of DDM.

\params
\verb$--prefixedorder$, \verb$--noprefixedorder$ \\
  If options \verb$--mincutoff$, \verb$--maxcutoff$, \verb$--minpairs$, \verb$--maxpairs$ are chosen then
  only \verb$--noprefixedorder$ is possible. 
  In other cases both options are available; the default one is \verb$--prefixedorder$.

\params \{\verb$--graphadj$\}, \verb$--nographadj$ \\
  These affect the way of determining adjacent vectors.

\params \{\verb$--plusplus$\}, \verb$--noplusplus$ \\
  If option \verb$--plusplus$ is chosen then only the pairs of adjacent vectors 
  that will be necessary on the future iterations are constructed. If option \verb$--noplusplus$ is chosen 
  then all edges are constructed on each iteration.

\params \{\verb$--bigint$\}, \verb$--int$, \verb$--float$ \\
  By default, arbitrary precision integer arithmetic is used.
  Option \verb$--int$ forces to use ordinary ($4$ bytes) integer precision arithmetic.
  Option \verb$--float$ forces to use double floating point ($8$ bytes) arithmetic.
  Option \verb$--rational$ forces to use rational arithmetic when exact enumerator / exact denominator pair 
         is used to represent rational number.

\params \verb$--zerotol value$ \\
  The option affects only if option \verb$--float$ is used.
  This is used to change a zero tolerance for floating point computation.
  A real value is considered as zero if its absolute value is at most the tolerance. 
  The default value for the zero tolerance is \verb$1e-8$.

\params \verb$--edges$,  \{\verb$--noedges$\} \\
  Option \verb$--edges$ forces to find edges, i.e. all pairs of adjacent vectors 
  in the ossature.

\params \verb$--adjacency$,  \{\verb$--noadjacency$\} \\
  Option \verb$--adjacency$ forces to find the lists of ossature vectors adjacent to each one.
  
\params \verb$--ridges$,  \{\verb$--noridges$\} \\ 
  Option \verb$--ridges$ forces to find ridges, i.e. all pairs of adjacent facets.

\params \verb$--facetadjacency$,  \{\verb$--nofacetadjacency$\} \\
  Option \verb$--facetadjacency$ forces to find the lists of facets adjacent to each one.

\params \verb$--verifyine$,  \{\verb$--noverifyine$\} \\
  Option \verb$--verifyine$ forces to determine implicit equations and redundand inequalities in the input system.

\params \verb$--inputfile filename$ \\
  This option defines input file name. \verb$skeleton --inputfile filename$
  is equivalent to \verb$skeleton filename$.

\params \verb$-o filename$ \verb$--outputfile filename$ \\
  This option sets the name of output file.
  By default, this name is obtained by adding
  extention \verb$.out$ to input file name.
  If input was from stdin then output file is \verb$skeleton.out$.
 
\params \verb$--logfile filename$ \\
  This option sets the name of log file.
  By default, this name is obtained by adding
  extention \verb$.log$ to input file name.
  If input was from stdin then log file is \verb$skeleton.log$.

\params \verb$--summaryfile filename$ \\
  This option sets the name of summary file.
  By default, this name is obtained by adding
  extention \verb$.sum$ to input file name.
  If input was from stdin then log file is \verb$skeleton.sum$.

\params \verb$--inputfromstdin$, \{\verb$--noinputfromstdin$\} \\
  \verb$--inputfromstdin$ forces to read input information fron stdin instead of file.
 
\params \{\verb$--simpleformat$\}, \verb$--skeletonformat$, \verb$--avisfukudaformat$ \\
  \verb$--skeletonformat$ indicates that input containes both inequalities and equations 
                          (both $A$ and $B$ matrices).\\
  \verb$--avisfukudaformat$ indicates that input is in Avis--Fukuda format (see \cite{Fukuda02});
                            all options in the file are ignored; only matrix in \verb$begin$--\verb$end$ 
                            parentheses is read; the number type specificator 
                            (\verb$integer$, \verb$ration$ etc.) is ignored.
                             
\params \{\verb$--outputinfile$\}, \verb$--nooutputinfile$ \\
  If \verb$--nooutputinfile$ is chosen then \Skeleton will not put results in output file. 

\params \verb$--outputonstdout$, \{\verb$--nooutputonstdout$\} \\
  If \verb$--outputonstdin$ is chosen then \Skeleton will put results on stdout. 

\params \{\verb$--loginfile$\}, \verb$--nologinfile$ \\
  If \verb$--nologinfile$ is chosen then \Skeleton will not put log information in log file. 

\params \{\verb$--logonstdout$\}, \verb$--nologonstdout$ \\
  If \verb$--nologonstdout$ is chosen then \Skeleton will not put 
  log information on stdout. 

\params \{\verb$--summaryinfile$\}, \verb$--nosummaryinfile$ \\
  If \verb$--nosummaryinfile$ is chosen then \Skeleton will not put 
  summary information in summary file. 

\params \{\verb$--summaryonstdout$\}, \verb$--nosummaryonstdout$ \\
  If \verb$--nosummaryonstdout$ is chosen then \Skeleton will not put 
  summary information on stdout. 
  
\params \verb$--silence$ \\
  This option is equivalent to \verb$--nooutputonstdout$, \verb$--nologonstdout$, \verb$--nosummaryonstdout$.

\params \verb$--ine$, \{\verb$--noine$\} \\
  If \verb$--ine$ is chosen then \Skeleton will put 
  the input matrix $A$ (coefficients of linear inequalities) on stdout or/and in outputfile. 
  This works only if option 
  \verb$--outputinfile$ or \verb$--outputonstdout$ correspondingly turns on.

\params \verb$--equ$, \{\verb$--noequ$\} \\
  If \verb$--equ$ is chosen then \Skeleton will put 
  the input matrix $B$ (coefficients of linear equations) on stdout or/and in outputfile. 
  This works only if option 
  \verb$--outputinfile$ or \verb$--outputonstdout$ correspondingly turns on.

\params \{\verb$--ext$\}, \verb$--noext$ \\ 
  If \verb$--noext$ is chosen then \Skeleton will not put 
  the matrix $V$ (with entries of ossature vectors) on stdout and in outputfile. 
  This works only if option 
  \verb$--outputinfile$ or \verb$--outputonstdout$ correspondingly turns on.

\params \{\verb$--bas$\}, \verb$--nobas$ \\
  If \verb$--nobas$ is chosen then \Skeleton will not put 
  the matrix $U$ (with entries of basis of maximal subspace contained in the cone) 
  on stdout and in outputfile. 
  This works only if option 
  \verb$--outputinfile$ or \verb$--outputonstdout$ correspondingly turns on.

\params \verb$--dis$, \{\verb$--nodis$\} \\
  If \verb$--dis$ is chosen then \Skeleton will put 
  the discrepancies matrix $VA\transpose$ on stdout or/and in outputfile. 
  This works only if option 
  \verb$--outputinfile$ or \verb$--outputonstdout$ correspondingly turns on.

\params \verb$--extinc$, \{\verb$--noextinc$\} \\
  If \verb$--extinc$ is chosen then for each vector in ossature
  the program will print (on stdout or/and in outputfile)
  iniqualities which hold as equality. 
  This works only if option 
  \verb$--outputinfile$ or \verb$--outputonstdout$ correspondingly turns on.

\params \verb$--ineinc$, \{\verb$--noineinc$\} \\
  If \verb$--ineinc$ is chosen then for each inequality in the initial system
  the program will print (on stdout or/and in outputfile)
  vectors in the ossature for which the inequality holds as equality. 
  This works only if option 
  \verb$--outputinfile$ or \verb$--outputonstdout$ correspondingly turns on.

\params \verb$--matrices$, \verb$--nomatrices$ \\
  \verb$--matrices$ is equivalent to 
  \verb$--ine$, \verb$--ext$, \verb$--bas$, \verb$--inc$;
  \verb$--nomatrices$ is equivalent to 
  \verb$--noine$, \verb$--noext$, \verb$--nobas$, \verb$--noinc$. 
  This works only if option 
  \verb$--outputinfile$ or \verb$--outputonstdout$ turns on.
 
\params \{\verb$--log$\}, \verb$--nolog$ \\
  If \verb$--nolog$ is chosen no log information 
  will not put on stdout and in log file. 
  This works only if option 
  \verb$--summaryinfile$ or \verb$--summaryonstdout$ turns on.

\params \{\verb$--summary$\}, \verb$--nosummary$ \\
  If \verb$--nosummary$ is chosen no summary information 
  (input/output/log file names, sizes of matrices and option values) 
  will not put on stdout and in summary file. 
  This works only if option 
  \verb$--summaryinfile$ or \verb$--summaryonstdout$ turns on.

\params \verb$-h$ or \verb$--help$ \\            
  \verb$skeleton -h$ prints the list of available options and terminates the program.
  \verb$skeleton --help$ does the same.

\params \verb$-v$ or \verb$--version$ \\      
  \verb$skeleton -v$   prints \Skeleton version and terminates the program.\\
  \verb$skeleton --version$ does the same.

\params \verb$--copying$ \\      
  \verb$skeleton --copying$ prints copyright info. 



\section{More Examples}

\subsection{Cube With a Cutted Vertex}

Consider the polyhedron described by the following system:
$$
\left\{
\begin{narrowarray}{rcrcrcc}
 x_1 &   &      &   &      & \ge & 0, \\
     &   &  x_2 &   &      & \ge & 0, \\
     &   &      &   &  x_3 & \ge & 0, \\
 x_1 &   &      &   &      & \le & 1, \\
     &   &  x_2 &   &      & \le & 1, \\
     &   &      &   &  x_3 & \le & 1, \\
2x_1 & + & 2x_2 & + & 2x_3 & \le & 5. \\
\end{narrowarray}
\right.
\tag{ex_cwcv_ine}
$$
It is a cube with a ``cutted'' vertex (see Fig.\,\ref{fig_cwcv}).

\begin{figure}
\centering
\psset{xunit=.7cm,yunit=.7cm}
  \begin{pspicture}(-2,-4)(6,6)
    \psline[arrows=->](-1.5,-1.5)(-2.25,-2.25)
    \psline[arrows=->](4,0)(5.5,0)
    \psline[arrows=->](0,4)(0,5.5)
    \rput[rt](-2.25,-2.25){$x_1$}
    \rput[lb](5.5,0){$~x_2$}
    \rput[rb](0,5.5){$x_3~$}

    \psline[linestyle=dashed](0,0)(4,0)
    \psline(-1.5,-1.5)(2.5,-1.5)
    \psline(0,4)(4,4)
    \psline(-1.5,2.5)(0.5,2.5)
    \psline[linestyle=dashed](0,0)(-1.5,-1.5)
    \psline(4,0)(2.5,-1.5)
    \psline(0,4)(-1.5,2.5)
    \psline(4,4)(3.25,3.25)
    \psline[linestyle=dashed](0,0)(0,4)
    \psline(4,0)(4,4)
    \psline(-1.5,-1.5)(-1.5,2.5)
    \psline(2.5,-1.5)(2.5,0.5)
    % triangle:
    \psline(0.5,2.5)(2.5,0.5)
    \psline(2.5,0.5)(3.25,3.25)
    \psline(3.25,3.25)(0.5,2.5)

    \psline(-2.5,1.25)(-1.5,1.25)
    \psline[arrows=->,linestyle=dashed](-1.5,1.25)(-0.75,1.25)
    \rput[r](-2.5,1.25){$x_2\ge 0~$}

    \psline(1.25,-2.3)(1.25,-1.5)
    \psline[arrows=->,linestyle=dashed](1.25,-1.5)(1.25,-0.75)
    \rput[t](1.25,-2.5){$x_3\ge 0~$}

    \psline[arrows=->](5,1.25)(3.25,1.25)
    \rput[l](5,1.25){$~x_2\le 1$}

    \psline[arrows=->](5,2.25)(2.25,2)
    \rput[l](5,2.25){$~2x_1+2x_2+2x_3\le 5$}

    \psline[arrows=->](1,5.2)(1,3.25)
    \rput[b](1,5.5){$~~x_3\le 1$}
 
    \psline[arrows=->](-2,-.5)(0.5,0.5)
    \rput[r](-2,-.5){$x_1\le 1~$}
 
    \psline(4.7,5.2)(3.5,4)
    \psline[arrows=->,linestyle=dashed](3.5,4)(1.75,2.25)
    \rput[b](4.75,5.5){$~~x_1\ge 0$}
 
    \rput[rb](-1.5,-1.5){$1~$}
    \rput[rb](0,4){$1~$}
    \rput[lb](4,0.25){$~1$}


  \end{pspicture}
\caption{Cube with a cutted vertex. Facet representation} \label{fig_cwcv}
\end{figure}

\begin{figure}
\centering
\psset{xunit=.7cm,yunit=.7cm}
  \begin{pspicture}(-2,-2.5)(6,6)
    \psline[arrows=->](-1.5,-1.5)(-2.25,-2.25)
    \psline[arrows=->](4,0)(5.5,0)
    \psline[arrows=->](0,4)(0,5.5)
    \rput[rt](-2.25,-2.25){$x_1$}
    \rput[lb](5.5,0){$~x_2$}
    \rput[lb](0,5.5){$~x_3$}

    \psline[linestyle=dashed](0,0)(4,0)
    \psline(-1.5,-1.5)(2.5,-1.5)
    \psline(0,4)(4,4)
    \psline(-1.5,2.5)(0.5,2.5)
    \psline[linestyle=dashed](0,0)(-1.5,-1.5)
    \psline(4,0)(2.5,-1.5)
    \psline(0,4)(-1.5,2.5)
    \psline(4,4)(3.25,3.25)
    \psline[linestyle=dashed](0,0)(0,4)
    \psline(4,0)(4,4)
    \psline(-1.5,-1.5)(-1.5,2.5)
    \psline(2.5,-1.5)(2.5,0.5)
    % triangle:
    \psline(0.5,2.5)(2.5,0.5)
    \psline(2.5,0.5)(3.25,3.25)
    \psline(3.25,3.25)(0.5,2.5)

    \psdots(0,0)
    \psdots(-1.5,-1.5)
    \psdots(0,4)
    \psdots(-1.5,2.5)
    \psdots(4,0)
    \psdots(4,4)
    \psdots(2.5,-1.5)
    \psdots(0.5,2.5)
    \psdots(2.5,0.5)
    \psdots(3.25,3.25)

    \rput[rb](-1.5,-1.5){$v_1~$}
    \rput[lt](2.5,-1.5){$~v_2$}
    \rput[l](2.5,0.5){$~v_3$}
    \rput[rb](-1.5,2.5){$v_4~$}
    \rput[b](0.5,2.65){$v_5~$}
    \rput[rb](0,4){$v_6~$}
    \rput[lb](4,4){$~v_7$}
    \rput[rb](3.25,3.25){$v_8~$}
    \rput[lt](4,-0.25){$v_9$}
    \rput[rb](0,0){$v_{10}~$}


  \end{pspicture}
\caption{Cube with a cutted vertex. Vertex representation} \label{fig_cwcv_v}
\end{figure}


The corresponding cone is described by the following homogenious system:
$$
\left\{
\begin{narrowarray}{rcrcrcrcl}
  x_1 &   &     &   &     &   &      & \ge & 0, \\
      &   & x_2 &   &     &   &      & \ge & 0, \\
      &   &     &   & x_3 &   &      & \ge & 0, \\
 -x_1 &   &     &   &     & + &  x_4 & \ge & 0, \\
      & - & x_2 &   &     & + &  x_4 & \ge & 0, \\
      &   &     & - & x_3 & + &  x_4 & \ge & 0, \\
-2x_1 & - &2x_2 & - &2x_3 & + & 5x_4 & \ge & 0, \\
      &   &     &   &     &   &  x_4 & \ge & 0  \\
\end{narrowarray}
\right.
\tag{ex_cwcv_ine_homo}
$$
(setting $x_4 = 1$ we get the initial system). So, input file 
(named \verb$cwcv.ine$) is
\verbatiminput{../examples/cwcv.ine}

Running \Skeleton with
\begin{verbatim}
skeleton cwcv.ine --edges --ineinc
\end{verbatim}
we get the following file \verb$cwcv.ine.out$:
\verbatiminput{../examples/cwcv.ine.out}

Matrix with entries of the basis is empty (it contains $0$ rows), hence
the polyhedron does not contain any non-zero linear variety.
Matrix with entries of the ossature has $10$ rows, hence the polyhedron
has $10$ vertex (see Fig.\,\ref{fig_cwcv_v}). The forth coordinate 
corresponds to the denominator in entries
of all these vertices. They are
$v_1=(1, 0, 0)\transpose$, 
$v_2=(1, 1, 0)\transpose$, 
$v_3=(1, 1, \frac{1}{2})\transpose$, 
$v_4=(1, 0, 1)\transpose$, 
$v_5=(1, \frac{1}{2}, 1)\transpose$, 
$v_6=(0, 0, 1)\transpose$, 
$v_7=(0, 1, 1)\transpose$, 
$v_{10}=\left(\frac{1}{2}, 1, 1\right)\transpose$,
$v_1=(0, 1, 0)\transpose$, 
$v_1=(0, 0, 0)\transpose$.

Also, we have computed all edges, i.e. pairs of adjacent vertices.
The polyhedron has $15$ edges. They are
$v_1$--$v_2$,
$v_1$--$v_4$,
$v_1$--$v_{10}$,
$v_2$--$v_3$,
$v_2$--$v_9$,
$v_3$--$v_5$,
$v_3$--$v_8$,
$v_4$--$v_5$,
$v_4$--$v_6$,
$v_5$--$v_8$,
$v_6$--$v_7$,
$v_6$--$v_{10}$,
$v_7$--$v_8$,
$v_7$--$v_9$,
$v_9$--$v_{10}$.

Information concerning ``Inequalities-to-rays incidence'' tell us that $7$ facets
are formed by vertices 
$v_6$, $v_7$, $v_9$, $v_{10}$;
$v_1$, $v_4$, $v_6$, $v_{10}$;
$v_1$, $v_2$, $v_9$, $v_{10}$;
$v_1$, $v_2$, $v_3$, $v_4$, $v_5$;
$v_2$, $v_3$, $v_7$, $v_8$, $v_9$;
$v_4$, $v_5$, $v_6$, $v_7$, $v_8$;
$v_3$, $v_5$, $v_8$
correspondingly.

Now we can check our computations by ``reversing'' them.
Form the input file (named \verb$cwcv.ext$) containing
entries of vertices found:
\verbatiminput{../examples/cwcv.ext}
and evoke \Skeleton:
\begin{verbatim}
skeleton cwcv.ext
\end{verbatim}
We get the following file \verb$cwcv.ext.out$:
\verbatiminput{../examples/cwcv.ext.out}
Since the matrix with ``basis'' is empty the polyhedron has full dimension.
The matrix with ``ossature'' has $7$ rows. They correspond to exactly the same
inequalities as in \Ref{ex_cwcv_ine}, so the polyhedron has $7$ facets.
We remark that in the list obtained there is no row corresponding to the inequality
$x_4 \ge 0$ because in our case it is redundant in \Ref{ex_cwcv_ine_homo}.

%Now consider a facet enumeration problem.
%Suppose we have $8$ points (see Fig.\,\ref{fig_Klee_Minty}):
%$(0,  0,   0)\transpose$,
%$(5,  0,   0)\transpose$, 
%$(5,  5,   0)\transpose$, 
%$(0, 25,   0)\transpose$, 
%$(0, 25,  25)\transpose$, 
%$(5,  5,  65)\transpose$, 
%$(5,  0,  85)\transpose$, 
%$(0,  0, 125)\transpose$. 
%Let's describe its convex hull.
%Prepare the input file \verb$kleeminty.ext$:
%\verbatiminput{../examples/kleeminty.ext}
%and run \Skeleton:
%\begin{verbatim}
%skeleton kleeminty.ext -noinc
%\end{verbatim}
%
%\begin{figure}
%\centering
%\psset{xunit=0.15cm,yunit=0.2cm}
%  \begin{pspicture}(-1,-1)(34,28)
%    \psline(0,7)(4,0)
%    \psline(4,0)(10,0)
%    \psline(10,0)(34,7)
%    \psline(34,7)(34,11)
%    \psline(34,11)(10,11)
%    \psline(10,11)(4,14)
%    \psline(4,14)(0,27)
%    \psline(0,27)(0,7)
%    \psline(4,0)(4,14)
%    \psline(10,0)(10,11)
%    \psline(0,27)(34,11)
%    \psline[linestyle=dashed](0,7)(34,7)
%    \rput[r](0,7){$(0,0,0)~~$}
%    \rput[tr](4,0){$(5,0,0)~$}
%    \rput[tl](10,0){$~(5,5,0)$}
%    \rput[tl](34,7){$~(0,25,0)$}
%    \rput[bl](34,11){$~(0,25,25)$}
%    \rput[tl](10,10){$~(5,5,65)$}
%    \rput[bl](4,14){$~(5,0,85)~$}
%    \rput[r](0,27){$(0,0,125)~~$}
%    \psdots(0,7)
%    \psdots(4,0)
%    \psdots(10,0)
%    \psdots(34,7)
%    \psdots(34,11)
%    \psdots(10,11)
%    \psdots(4,14)
%    \psdots(0,27)
%  \end{pspicture}
%\caption{} \label{fig_Klee_Minty}
%\end{figure}
%
%We'll get the following output file \verb$kleeminty.ext$:
%\verbatiminput{../examples/kleeminty.ext.out}
%Since the matrix with ``basis'' is empty the polyhedron has full dimension.
%The matrix with ``ossature'' has $6$ rows, so the polyhedron has $6$ faces and can be
%describe by the following system of linear inequalities:
%$$
%\left\{
%\begin{narrowarray}{rrrcr}
%      &       &  x_3 & \ge &   0,\\
%-4x_1 &  -x_2 &      & \ge & -25,\\ 
% -x_1 &       &      & \ge &  -5,\\ 
%  x_1 &       &      & \ge &   0,\\ 
%      &   x_2 &      & \ge &   0,\\ 
%-8x_1 & -4x_2 & -x_3 & \ge & 125. 
%\end{narrowarray}
%\right.
%$$


\subsection{Implicit equations and redundant inequalities}

\Skeleton can find implicit equations and redundant inequalities in a system.
Let's consider the system
$$
\left\{
\begin{narrowarray}{rcrcrcrcr}
 x_1 &   &      &   &     &   &       &  \ge & 0, \\
     &   &  x_2 &   &     &   &       &  \ge & 0, \\
     &   &      &   & x_3 &   &       &  \ge & 0, \\
     &   &      &   &     &   &  x_4  &  \ge & 0, \\
 x_1 &  +& 2x_2 &   &     &  +& 3x_4  &  \ge & 0, \\
 x_1 &  +&  x_2 &  +& x_3 &  +& 3x_4  &  \ge & 0, \\
 x_1 &  -& 2x_2 &  +& x_3 &  +& 3x_4  &  \ge & 0, \\
-x_1 &  +& 2x_2 &  -& x_3 &  -& 3x_4  &  \ge & 0. \\
\end{narrowarray}
\right.
$$
So, the input file (named \verb$equ.ine$) is
\verbatiminput{../examples/equ.ine}

Running \Skeleton with
\begin{verbatim}
skeleton equ.ine --verifyine
\end{verbatim}
we get the following file \verb$equ.ine.out$:
\verbatiminput{../examples/equ.ine.out}

This mean that two inequalities in the original system , the $7$th and the $8$th, are implicit equations;
the $2$nd, $5$th and $6$th inequalities are redundant.

\subsection{Skeleton ``extended'' format}

\Skeleton can treat systems containing both inequalities and equations (explicitly defined):
both $A$ and $B$ matrices. For this it is necessary to use a special format in the input file. 
Here is an example:
$$
\left\{
\begin{narrowarray}{rcrcrcrcr}
 x_1 &  -&  x_2 &  -& x_3 &  =  & 0, \\
 x_1 &   &      &   &     & \ge & 0, \\
     &   &  x_2 &   &     & \ge & 0, \\
     &   &      &   & x_3 & \ge & 0. \\
\end{narrowarray}
\right.
$$
The input file (names \verb$sf.ine$) follows. 
\verbatiminput{../examples/sf.ine}
Running \Skeleton with
\begin{verbatim}
skeleton sf.ine --skeletonformat
\end{verbatim}
we get the following file \verb$sf.ine.out$:
\verbatiminput{../examples/sf.ine.out}

\subsection{Avis--Fukuda format}

\Skeleton partially supports Avis--Fukuda format (see \cite{Avis,Fukuda02}).
All options (except a matrix inside \verb$begin$--\verb$end$ parentheses) in the file are ignored.
The number type specificator (\verb$integer$, \verb$ration$ etc.) is also ignored.
Note that the most of options in Avis--Fukuda format has equivalent ones in \Skeleton 
but they must be indicated in command line.

%Let's consider, for example, the file \verb$cyc.ine$ taken from D.\,Avis \verb$lrs$ repository:
Let's consider, for example, the file \verb$ucube.ine$ taken from K.\,Fukuda \verb$cdd$ repository \cite{Fukuda02}:
\verbatiminput{../examples/ucube.ine}
Run \Skeleton:
\begin{verbatim}
skeleton ucube.ine --avisfukudaformat --adjacency --facetadjacency 
                   --extinc --ineinc
\end{verbatim}
The output file is
\verbatiminput{../examples/ucube.ine.out}
The log file is
\verbatiminput{../examples/ucube.ine.log}

\subsection{Voronoi Diagram}

Let $W$ be a system of $s$ points in $\RR^d$. For each $w\in W$ we can
consider the set
$$
\VS(w) = \set{x\in\RR^d:~\forall v\in W \setminus\set{w}~\dist(x,p)\le\dist(x,q)},
$$
where $\dist$ is the Euclidean distance function.
The set $\VS(w)$ is called a {\em Voronoi cell}. It is a polyhedron.
Its vertices are called {\em Voronoi vertices} and extreme rays are called
{\em Voronoi rays}. The set $\set{\VS(w):~w\in W}$ of all Voronoi cells 
is called {\em Voronoi diagram} (see Fig.\,\ref{fig_voronoi_diag}). 
For generating Voronoi diagram the following
construction is widely used.

\begin{figure}
\centering
\psset{xunit=1.1cm,yunit=1.1cm}
  \begin{pspicture}(-3,-1)(3,4)
    \psdots(0,0)
    \psdots(2,0)
    \psdots(-2,0)
    \psdots(0,1)
    \psdots(1,2)
    \psdots(-1,2)
    \psdots(0,3)

    \rput[l](0,0){$~w_1$}
    \rput[l](2,0){$~w_2$}
    \rput[l](-2,0){$~w_3$}
    \rput[l](0,1){$~w_4$}
    \rput[l](1,2){$~w_5$}
    \rput[l](-1,2){$~w_6$}
    \rput[l](0,3){$~w_7$}


    \psline(-1,.5)(1,.5)
    \psline(-1,.5)(-1.1666,0.8333) 
    \psline(1,.5)(1.1666,0.8333)
    \psline(1.1666,0.8333)(0,2)
    \psline(-1.1666,0.8333)(0,2)

    \psline[arrows=->](-1,.5)(-1,-.5)
    \psline[arrows=->](1,.5)(1,-.5)
    \psline[arrows=->](1.1666,0.8333)(3.1666,1.8333)
    \psline[arrows=->](-1.1666,0.8333)(-3.1666,1.8333)
    \psline[arrows=->](0,2)(1,3)
    \psline[arrows=->](0,2)(-1,3)

  \end{pspicture}
\caption{Voronoi diagram for the set of points} \label{fig_voronoi_diag}
\end{figure}


For each $w\in W$ consider the hyperplane tangent at $w=(w_1,\dots,w_d)\transpose$ 
to the paraboloid 
$\set{(x_1,\dots,x_d,x_{d+1}):~x_{d+1} = x_1^2+\dots+x_d^2}$. This hyperplane is
represented by the following equation:
$$
-2w_1x_1-\dots-2w_dx_d+x_{d+1} +w_1^2+\dots+w_d^2= 0.
$$
Replacing the equality with inequality $\ge$ and considering these
inequalities for each $w\in W$ we get the system of $s$ linear
inequalities. Let $P$ be the polyhedron of all solutions to the system. It
turns out that $P$ is a lifting of Voronoi diagram to one higher
dimensional space; and the projection of each facet of $P$ associated with
$w$ is exactly the Voronoi cell $\VS(w)$. The vertices and extreme rays of
$P$ project exactly to the Voronoi vertices and rays, respectively \cite{Fukuda2004}.

As an example consider the set of points 
$(0,0)\transpose$, 
$(2,0)\transpose$, 
$(-2,0)\transpose$, 
$(0,1)\transpose$, 
$(1,2)\transpose$, 
$(-1,2)\transpose$, 
$(0,3)\transpose$.
For generating their Voronoi diagram 
consider the system
$$
\left\{
\begin{narrowarray}{rcrcrcrcl}
      &   &      & + & x_3 &   &      & \ge & 0, \\
-4x_1 &   &      & + & x_3 & + & 4x_4 & \ge & 0, \\
 4x_1 &   &      & + & x_3 & + & 4x_4 & \ge & 0, \\
      & - & 2x_2 & + & x_3 & + &  x_4 & \ge & 0, \\
-2x_1 & - & 4x_2 & + & x_3 & + & 5x_4 & \ge & 0, \\
 2x_1 & - & 4x_2 & + & x_3 & + & 5x_4 & \ge & 0, \\
      & - & 6x_2 & + & x_3 & + & 9x_4 & \ge & 0, \\
      &   &      &   &     &   &  x_4 & \ge & 0.
\end{narrowarray}
\right.
$$
Prepare file \verb$exvoronoi.ine$:
\verbatiminput{../examples/exvoronoi.ine}
Now evoke \Skeleton:
\begin{verbatim}
skeleton exvoronoi.ine --ineinc --extinc
\end{verbatim}
We get the following file \verb$exvoronoi.ine.out$:
\verbatiminput{../examples/exvoronoi.ine.out}
 
\begin{figure}
\centering
\psset{xunit=1.1cm,yunit=1.1cm}
  \begin{pspicture}(-3,-1)(3,4)
    \psdots(0,0)
    \psdots(2,0)
    \psdots(-2,0)
    \psdots(0,1)
    \psdots(1,2)
    \psdots(-1,2)
    \psdots(0,3)


    \psline(-1,.5)(1,.5)
    \psline(-1,.5)(-1.1666,0.8333) 
    \psline(1,.5)(1.1666,0.8333)
    \psline(1.1666,0.8333)(0,2)
    \psline(-1.1666,0.8333)(0,2)

    \psline[arrows=->](-1,.5)(-1,-.5)
    \psline[arrows=->](1,.5)(1,-.5)
    \psline[arrows=->](1.1666,0.8333)(3.1666,1.8333)
    \psline[arrows=->](-1.1666,0.8333)(-3.1666,1.8333)
    \psline[arrows=->](0,2)(1,3)
    \psline[arrows=->](0,2)(-1,3)

    \rput[t](-1,-.559){$v_1$}
    \rput[t](1,-.559){$v_1$}
    \rput[lb](1,3){$~v_2$}
    \rput[l](3.1666,1.8333){$~v_3$}
    \rput[rb](-1,3){$v_4~$}
    \rput[r](-3.1666,1.8333){$v_5~$}

    \psdots[dotstyle=o](-1,.5)
    \psdots[dotstyle=o](1,.5)
    \psdots[dotstyle=o](1.1666,0.8333)
    \psdots[dotstyle=o](-1.1666,0.8333)
    \psdots[dotstyle=o](0,2)

    \rput[rt](-1,.5){$v_6~$}
    \rput[lt](1,.5){$~v_7$}
    \rput[l](0,2){$~~v_8$}
    \rput[l](-1.1666,0.8333){$~~v_9$}
    \rput[r](1.1666,0.8333){$v_{10}~~$}



  \end{pspicture}
\caption{Voronoi diagram constructed with the help of \Skeleton} \label{fig_voronoi_diag_Skeleton}
\end{figure}
Each extreme ray with last entry equal to $0$ corresponds to a Voronoi ray. 
Each ray whose last entry is non-zero corresponds to a Voronoi vertex. 
So, we get $5$ Voronoi rays (ignoring the third component):
$$     
v_1 = (0, -1)\transpose,~      
v_2 = (1, 1)\transpose,~    
v_3 = (2, 1)\transpose,~    
v_4 = (-1, 1)\transpose,~    
v_5 = (-2, 1)\transpose.     
$$
and $5$ Voronoi vertices
(dividing by the forth component and ignoring the third one):
$$
v_6 = \left(-1,~ \frac12\right)\transpose,\quad   
v_7 = \left( 1,~ \frac12\right)\transpose,\quad  
v_8 = (0,~ 2)\transpose,
$$
$$
v_9 = \left(-\frac76,~ \frac56\right)\transpose,\quad 
v_{10} = \left(\frac76,~ \frac56\right)\transpose.
$$
Interpreting ``Edges'' or/and ``Inequalities-to-rays incidence'' we get Fig.\,\ref{fig_voronoi_diag_Skeleton}.



\subsection{Delaunay Triangulation}

                                                                          
                                                                          
Let $W$ be a system of $s$ points in $\RR^d$ and $v$ be some Voronoi
vertex for $W$. The convex hull of the nearest neighbor set of $v$ is
called the {\em Delaunay cell} of $v$. The {\em Delaunay complex} (or {\em
triangulation}) of $W$ is a partition of $\Conv W$ into the Delaunay cells
of Voronoi vertices.

The Delaunay complex is not in general a triangulation but becomes a
triangulation when the points in $W$ are in {\em general position} (or
{\em nondegenerate}), i.e. no $d+2$ points are cospherical or equivalently there
is no point $c \in \RR^d$ whose nearest neighbor set has more than $d + 1$
elements.

The Delaunay complex is dual to the Voronoi diagram in the sense that there is a
natural bijection between the two complexes which reverses the face inclusions
(see Fig.\,\ref{fig_delaunay})
\cite{Fukuda2004}.

\begin{figure}
\centering
\psset{xunit=1.1cm,yunit=1.1cm}
  \begin{pspicture}(-3,-1)(3,4)
    \psdots(0,0)
    \psdots(2,0)
    \psdots(-2,0)
    \psdots(0,1)
    \psdots(1,2)
    \psdots(-1,2)
    \psdots(0,3)

    \rput[t](0,-.15){$~w_1$}
    \rput[l](2,0){$~w_2$}
    \rput[r](-2,0){$w_3~$}
    \rput[l](0,1){$~~w_4$}
    \rput[l](1,2){$~w_5$}
    \rput[r](-1,2){$w_6~$}
    \rput[b](0,3.15){$w_7$}

    \psline(0,0)(2,0)
    \psline(0,0)(-2,0)
    \psline(0,0)(0,1)
    \psline(2,0)(0,1)
    \psline(-2,0)(0,1)
    \psline(2,0)(1,2)
    \psline(-2,0)(-1,2)
    \psline(0,1)(1,2)
    \psline(0,1)(-1,2)
    \psline(1,2)(0,3)
    \psline(-1,2)(0,3)

    \psline[linestyle=dashed](-1,.4)(1,.4)
    \psline[linestyle=dashed](-1,.4)(-1.1666,0.8333) 
    \psline[linestyle=dashed](1,.4)(1.1666,0.8333)
    \psline[linestyle=dashed](1.1666,0.8333)(0,2)
    \psline[linestyle=dashed](-1.1666,0.8333)(0,2)

    \psline[arrows=->,linestyle=dashed](-1,.4)(-1,-.5)
    \psline[arrows=->,linestyle=dashed](1,.4)(1,-.5)
    \psline[arrows=->,linestyle=dashed](1.1666,0.8333)(3.1666,1.8333)
    \psline[arrows=->,linestyle=dashed](-1.1666,0.8333)(-3.1666,1.8333)
    \psline[arrows=->,linestyle=dashed](0,2)(1,3)
    \psline[arrows=->,linestyle=dashed](0,2)(-1,3)

    \psdots[dotstyle=o](-1,.4)
    \psdots[dotstyle=o](1,.4)
    \psdots[dotstyle=o](1.1666,0.8333)
    \psdots[dotstyle=o](-1.1666,0.8333)
    \psdots[dotstyle=o](0,2)

  \end{pspicture}
\caption{Delaunay triangulation is dual to Voronoi diagram} \label{fig_delaunay}
\end{figure}


So, to generate Delaunay triangulation we can perform the following procedure.
For each vertex of polyhedra (we are not interesting in extreme rays) in previous section
we determine all facets incident to the vertex.
Interpreting information about ``Rays-to-inequalities incidence'' in \verb$exvoronoi.ine.out$
we get that Delaunay cells in this example are formed by the following
vertices 
$w_1$, $w_3$, $w_4$;
$w_1$, $w_2$, $w_4$;
$w_2$, $w_4$, $w_5$;
$w_3$, $w_4$, $w_6$;
$w_4$, $w_5$, $w_6$, $w_7$ (see Fig.\,\ref{fig_delaunay}).

There is a direct way to construct the Delaunay triangulation. Consider the
same paraboloid as in the previous section: $x_{d+1} = x_1^2+\dots+x_d^2$.
For each point $w=(w_1,\dots,w_d)\transpose$ in $W$ consider its lifting
$(w_1,\dots,w_d,w_1^2+\dots+w_d^2)\transpose$ in $\RR^{d+1}$ and take the
convex hull $P$ of all such lifted points. Let $v=(0,\dots,0,1)$. It turns
out that any facet of $P+\NonNeg(v)$ which is not parallel to $v$ is a
Delaunay cell once its last coordinate is ignored, and any Delaunay cell is
represented this way \cite{Fukuda2004}.

For our example form the file \verb$exdelaunay.ext$:
\verbatiminput{../examples/exdelaunay.ext}
and evoke \Skeleton:
\begin{verbatim}
skeleton exdelaunay.ext --extinc
\end{verbatim}
We get the file \verb$exdelaunay.ext.out$:
\verbatiminput{../examples/exdelaunay.ext.out}
Only first $5$ facets are not parallel to $v$ (because their 3rd coordinate is non-zero).
So, we again have $5$ Delaunay cells which are formed by points
$w_1$, $w_3$, $w_4$;
$w_1$, $w_2$, $w_4$;
$w_4$, $w_5$, $w_6$, $w_7$;
$w_3$, $w_4$, $w_6$;
$w_2$, $w_4$, $w_5$
correspondingly  (see Fig.\,\ref{fig_delaunay}).

\bibliographystyle{alpha}

\bibliography{skeleton}

\end{document}

